\section{Summary and Outlook}
The results reflect our expectation very well. At low pedestrian densities the simulation is comparable to the one of Wood and B�cheler with no pedestrians at all. As expected, roundabouts are much more sensitive to pedestrians than crossroads. Crossroads keep their functionality to very high pedestrian densities, whereas roundabouts collapse. So it's quite reasonable to use crossroads in cities. But the maximum traffic flow in roundabouts can be twice as high than in cross rounds. For highways outside cities, roundabouts can therefore be a good choice, especially because they are simple and they normally need more space than crossroads. So, as always in life, every system as its advantages under certain conditions. 


There are many possible modifications to develop in this simulation. As mentioned in the description, an intelligent control of the traffic light could may boost the efficiency of crossroads. Vice versa a traffic light at the entry of a roundabout kicked in at high pedestrian densities could improve the efficiency and avoid a collapse. It would be interesting to analyse and simulate different hybrid models, for example with a mixed city configuration of crossroads and roundabouts, or using these controlled roundabouts.  